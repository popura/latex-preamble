\usepackage{tcolorbox}
\tcbuselibrary{breakable, skins, theorems}
\usepackage{varwidth} 
\usepackage{marginnote}

\definecolor{frameinnercolor}{RGB}{49,44,44}

\newenvironment{theorem}[3][]{%
%#1 = タイトル, #2 = 環境名, #3 = 定理環境名
\refstepcounter{#2}%
\newtcolorbox{theobox}[1][]{%
enhanced,frame empty,interior empty,
coltitle=white,fonttitle=\bfseries,colbacktitle=frameinnercolor,
extras broken={frame empty,interior empty},
borderline={0.5mm}{0mm}{frameinnercolor},
sharp corners=downhill,
breakable=true,
top=4mm,
before skip=3.5mm,
attach boxed title to top left={yshift=-3mm,xshift=3mm},
boxed title style={boxrule=0pt,sharp corners=all},varwidth boxed title,
title=##1,
}
\ifstrempty{#1}{% ifstremptyはetoolbox.styで定義.etoolbox.styはtcolorboxが読み込むので宣言不要
\begin{theobox}[#3~\csname the#2\endcsname.]}
{\begin{theobox}[#3~\csname the#2\endcsname:~{#1}]}
}{\end{theobox}}

\newcounter{theorem}
\numberwithin{theorem}{section}% numberwithinはamsmath.styで定義されている
\newenvironment{theo}[1][]%
{\begin{theorem}[#1]{theorem}{定理}%
}{\end{theorem}}

\newcounter{proposition}
\numberwithin{proposition}{section}
\newenvironment{prop}[1][]%
{\begin{theorem}[#1]{proposition}{命題}%
}{\end{theorem}}

\newcounter{lemma}
\numberwithin{lemma}{section}
\newenvironment{lem}[1][]%
{\begin{theorem}[#1]{lemma}{補題}%
}{\end{theorem}}

\newcounter{example}
\numberwithin{example}{section}
\newenvironment{ex}[1][]%
{\begin{theorem}[#1]{example}{例}%
}{\end{theorem}}

\newcounter{exercise}
\numberwithin{exercise}{section}
\newenvironment{exc}[1][]%
{\begin{theorem}[#1]{exercise}{演習問題}%
}{\end{theorem}}

\newcounter{practice}
\numberwithin{practice}{section}
\newenvironment{prac}[1][]%
{\begin{theorem}[#1]{practice}{例題}%
}{\end{theorem}}


\newtcolorbox{board}[1][]{%
enhanced,frame empty,interior empty,
coltitle=white,fonttitle=\bfseries,colbacktitle=frameinnercolor,
extras broken={frame empty,interior empty},
borderline={0.5mm}{0mm}{frameinnercolor},
sharp corners=downhill,
breakable=true,
top=4mm,
before skip=3.5mm,
title={#1},
}
